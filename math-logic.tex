\documentclass[a4paper]{article}
\usepackage[utf8]{inputenc}

\usepackage[margin = 1in]{geometry}

\usepackage{amsmath}
\usepackage{amssymb}
\usepackage{amsthm}
\usepackage{enumerate}
\usepackage{graphicx}
\usepackage{listings}
\usepackage{color}


\definecolor{codegreen}{rgb}{0,0.6,0}
\definecolor{codegray}{rgb}{0.5,0.5,0.5}
\definecolor{codepurple}{rgb}{0.58,0,0.82}
\definecolor{backcolour}{rgb}{0.95,0.95,0.92}
 
\lstdefinestyle{mystyle}{
    backgroundcolor=\color{backcolour},   
    commentstyle=\color{codegreen},
    keywordstyle=\color{blue},
    numberstyle=\footnotesize\color{codegray},
    stringstyle=\small\color{red},
    basicstyle=\ttfamily,
    breakatwhitespace=false,         
    breaklines=true,                 
    captionpos=b,                    
    keepspaces=true,                 
    numbers=left,                    
    numbersep=5pt,                  
    showspaces=false,                
    showstringspaces=false,
    showtabs=false,                  
    tabsize=4
}

\lstset{style=mystyle}

\newcommand{\ldq}{``}
\newcommand{\lsq}{`}


\theoremstyle{definition}
\newtheorem{example}{Example}

\theoremstyle{definition}
\newtheorem{definition}{Definition}

\theoremstyle{definition}
\newtheorem{theorem}{Theorem}

\theoremstyle{definition}
\newtheorem{corol}{Corollary}

\theoremstyle{definition}
\newtheorem{prop}{Proposition}

\theoremstyle{definition}
\newtheorem{algo}{Algorithm}

\theoremstyle{definition}
\newtheorem{exer}{Exercise}

\parindent = 0in
\parskip = 6pt

\title{Foundations of Mathematical Logic}
\author{Muhammad Muaaz Tariq}
\date{\today}

\begin{document}
\maketitle

\section*{Introduction}
Logic is the base of any scientific theory or law. Mathematics, too, has very strong roots in reasoning. In these handouts, we will be exploring about how things are established in mathematics on the basis of logic and reasoning. We will start from the foundations.
 
\section{Propositions}
\textit{A declarative sentence which is either true or false, is called a proposition.} A proposition is a sentence about which we can decide if it is true or false. For example:
\begin{enumerate}[i.]
    \item Karachi is the capital of Pakistan.
    \item Cat is a mammal.
    \item I have not visited Norway.
    \item $2 + 3 = 5$
    \item $ 17 > 60$
\end{enumerate}
In the above sentences, you can see that (i) and (v) are false whereas, (ii) and (iv) are true. So these are propositions. What about (iii)? How can we know whether it is true or false? In fact, at the time this statement is made, we will know about its being true of false. If a person has visited Norway and he makes this statement, we will know that this statement is false. In the other case, if the person to say this never visited Norway, this sentence will be true.


The following sentences are not proposition.
\begin{enumerate}[i.]
    \item Where are you going?
    \item Please, do not go to Harry's saloon.
    \item Read this document carefully.
    \item $x \neq 9$
\end{enumerate}
Sentence (i) is a question. So, it is not about being true or false. Sentences (ii) and (iii) are instructions. Sentence (iv) is also not proposition as we do not know the value of $x$, even at the time of making this statement. We cannot decide about it unless we know the value of $x$. However, we can change it into a proposition by replacing $x$ with some number. For example, if we replace $x$ with 6, then this sentence will be false.

\subsection{Truth Value of A Propostion}
Consider these two propositions.
\begin{enumerate}[i.]
    \item Canada is an Asian country.
    \item $ 5 + 8 > 10 $
\end{enumerate}
As expected, one of them, (ii), is true and the other, (i), is fasle. \textit{Being true or false of a proposition is called its truth value.} A propsition which is true has the truth value \textit{true}, and \textit{false} otherwise. Th truth values $true$ and $false$ are represented by $T$ and $F$, respectively. In the above sentences, the truth value of (i) is $false$ or $F$, and truth value of (ii) is $true$ or $T$.

\subsection{Propositional Variables}
We can use variables to represent propositions (propositional sentences). For example, the proposition ``Marry got a job'' can be represented by a variable $x$. Now, $x$ is a proposition its truth value depends upon the truth value of ``Marry got a job''. Suppose, ``The sun sets in the west'' is represented by the variable $A$. We will say, proposition $A$ is true, or the truth value of $A$ is $T$.

We use alphabets to name the propositional variables, but in these handouts, we will be using the names like, $p, q, r$, etc.

\begin{example}
    Use variables to represent the following propositions.
    \begin{enumerate}[i.]
        \item I do not like coffee.
        \item Johnny is going to a trip tomorrow.
        \item We are junior students.
    \end{enumerate}
    \textbf{Solution:} $p$ represents (i), $q$ represents (ii), and $r$ represents (iii). That's it. As an additional note, if Johnny does not go to trip tomorrow, the truth value of $q$ will be $false$ or $F$.
\end{example}

\end{document}